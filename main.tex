\documentclass[a4paper]{article}

\usepackage{amsmath}
\usepackage{amsfonts}
\usepackage{amssymb}
\usepackage{graphicx}
\usepackage[spanish]{babel}
\usepackage[utf8]{inputenc}
\usepackage{hyperref}
\usepackage{float}

\usepackage{authblk}

\title{Clase Oscilador y gráficas.}

% Author names and affiliations
\author{Verónica Cardona Gómez.}




\affil[*]{\underline{veronica.cardonag1@udea.edu.co}}



\begin{document}
\maketitle

\section{Osciladores armónicos simples.}

\begin{figure}[H]
\centering
\includegraphics[width=0.8\textwidth]{arms.jpg}
\caption{\label{fig1} y vs t para cinco osciladores armónicos simples.}
\end{figure}

En la figura 1 podemos observar la posición vs el tiempo de un oscilador armónico simple, donde se preserva la amplitud de la función, lo cual nos lleva a concluir que no hay pérdida de energía y que hay coherencia con los parámetros ingresados debido a que gamma, el coeficiente de amortiguamiento es cero.


\begin{figure}[H]
\centering
\includegraphics[width=0.8\textwidth]{armsfase.jpg}
\caption{\label{fig1} v vs y (fase) para cinco osciladores armónicos simples.}
\end{figure}

En la figura 2 vemos la fase para los osciladores armónicos simples, donde se aprecia que el sistema oscilante obtiene su máxima velocidad en el origen y al llegar a sus amplitudes máximas, es cero.

\section{Osciladores amortiguados.}

\begin{figure}[H]
\centering
\includegraphics[width=0.8\textwidth]{amort.jpg}
\caption{\label{fig1} y vs t para cinco osciladores amortiguados.}
\end{figure}

En la figura 3 podemos observar la posición vs el tiempo de un oscilador amortiguado, donde se disminuye la amplitud de la función progresivamente, significando una pérdida de energía debido a la presencia de una fuerza viscosa o de amortiguamiento, esto concuerda con los parámetros ingresados, en los cuales se consideró amortiguamiento débil ($\frac{\gamma}{2}$ menor que $\frac{k}{m}$).

\begin{figure}[H]
\centering
\includegraphics[width=0.8\textwidth]{amortfase.jpg}
\caption{\label{fig1} v vs y (fase) para cinco osciladores amortiguados.}
\end{figure}

Por último, en la figura 4 vemos la fase para los osciladores amortiguados, donde es evidente la disminución sucesiva en el tiempo de la amplitud y la velocidad del sistema oscilante, ocurre una situación similar, aunque no igual, a la de la fase de los osciladores armónicos simples, cuando pasa por el orígen, tendrá una velocidad mayor que cuando llega a los extremos o por decirlo así, a las amplitudes máximas momentáneas.


\end{document}
